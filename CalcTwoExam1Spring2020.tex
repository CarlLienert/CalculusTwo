
\documentclass[12pt]{article}
%\usepackage{amsmath,array,latexsym}
\usepackage{amsmath}
\usepackage{geometry}                % See geometry.pdf to learn the layout options. There are lots.
\geometry{letterpaper}                   % ... or a4paper or a5paper or ... 
%\geometry{landscape}                % Activate for for rotated page geometry
%\usepackage[parfill]{parskip}    % Activate to begin paragraphs with an empty line rather than an indent
\usepackage{graphicx}
\usepackage{amssymb}
\usepackage{epstopdf}
\usepackage{amsthm}
\usepackage{color}
\DeclareGraphicsRule{.tif}{png}{.png}{`convert #1 `dirname #1`/`basename #1 .tif`.png}
\usepackage{epstopdf}
\usepackage{graphicx}
\setlength{\oddsidemargin}{0pt}
\setlength{\evensidemargin}{0pt}
\setlength{\topmargin}{0pt}
\setlength{\marginparwidth}{0pt}
\setlength{\textwidth}{6.5in}
\setlength{\textheight}{9in}
\setlength{\headheight}{0pt}
\setlength{\headsep}{0pt}
\newcommand{\pic}{pix}
\newcommand{\ben}{\begin{enumerate}}
\newcommand{\een}{\end{enumerate}}
\newcommand{\fig}[4][\pic]{\resizebox{#2}{#3}{\includegraphics{#1/#4}}}
\begin{document}


\begin{center}
\textbf{Calculus II, Exam One}
\end{center}
\bigskip

\noindent5 - February, 2020\hspace*{\fill} Name: \hrulefill\\ \\
{\bf Instructions:} You may not use a calculator or other electronic/internet computation device.  Show all your work.\\
\begin{enumerate}
\item Evaluate the following integrals. 
\begin{enumerate}
%\item $$\int \frac{x^7+5x^3+1}{x} \ dx$$ \pagebreak
%\item $$\int_{-1}^1 x^2 e^{\left(x^3+1\right)} \ dx$$ \pagebreak
%\item $$\int_0^\infty x e^{-x} \ dx$$ \pagebreak
%\item  $\displaystyle \int \frac{x^2}{ \sqrt{4-x^2}}\ dx$\ \  (Hint: $\frac{d}{dx} \sec \theta = \sec \theta \tan \theta$)\\
%
%fixed: kind of: but some might try partial fractions:
% $\displaystyle \int \frac{x^2}{ 4-x^2}\ dx$\\
% 
% or, for a nice integrand after substitution:  
 \item $\displaystyle \int \frac{1}{(x^2+4)^{3/2}}\ dx$\\

\pagebreak

%\item $$ \int \frac{4-2x+3x^2}{2x^2-x^3}\ dx$$ \pagebreak
%\item $$\int_1^\infty \frac{x}{x^2+4}\ dx$$ or $$\int_1^\infty \frac{1}{x^2+4}\ dx$$ % \pagebreak
%\item $$\int x^2 \ln x \ dx$$

\item $\displaystyle \int \frac{2}{(x+1)(x-3)} \ dx$ \\ \\
\pagebreak
\item $\displaystyle \int_0^1 \frac{e^{2y}}{e^{2y}+1} \ dy$  \\ \\
\pagebreak
\item  $\displaystyle \int_1^2 x^2\ln x\ dx$ 
\end{enumerate}
\pagebreak
\item Evaluate the following improper integral.  You must use correct notation, including the appropriate limit notation.\\

$\displaystyle \int_0^\infty e^{-2x}\ dx$
\pagebreak

%\begin{enumerate}
%\item $$\int t e^{\left(t^2+1\right)} \ dt$$
%\item $$\int_1^2 x^3 \ln x \ dx$$
%\item $$\displaystyle \int \frac{3x+1}{x^2+x} \ dx$$
%\item $$\displaystyle \int \frac{1}{\sqrt{1-9x^2}} \ dx$$
%\item $$\int_0^5 \frac{1}{\sqrt{25-x^2}}\ dx$$
%\item $$\int_1^\infty \frac{1}{x^3+x}\ dx$$
%\item $$\int \frac{1}{x^2+4x+5}\ dx$$
%\item $$ \int \frac{1+x^2}{x(1+x)^2}\ dx$$
%\end{enumerate}
%\item Use the comparison test to determine whether the following improper integral converges or diverges.  You must show all your work, and give a complete argument.\\
%
%$\displaystyle \int_2^\infty \frac{1}{\sqrt{x}-1}\ dx$ 
%
%replace with?:  $\displaystyle \int_1^2 x^2\ln x\ dx$ 

\pagebreak


%\item During a surge in demand for electricity, the rate, $r$, at which energy is used can be approximated by 
%$$r = te^{-at}$$
%where $t$ is the time in hours and $a$ is a positive constant.
%
%\begin{enumerate}
%
%\item Find the total energy, $E$, used in the first $T$ hours.  Give your answer as a function of $a$. \\ \\
%\vspace{10cm}
%\item What happens to $E$ as $T \rightarrow \infty$?
%
%\end{enumerate}
%\pagebreak
%
%\item Write the Riemann sum and associated definite integral using the indicated slice to find the volume of the solid below. Evaluate the integral.\\
%\begin{center}
% \includegraphics[width=6cm]{816}
% \end{center}
\item Sketch a graph of a function, $f(x)$, and indicate the values $a$ and $b$ on the $x$-axis that satisfy the given description.
\begin{enumerate}
\item $\displaystyle \int_a^b f(x) \ dx$ is underestimated by a left sum, and overestimated by a trapezoidal sum. Sketch your function twice, on one set of axes sketch rectangles that show the left sum is an underestimate. On the other set of axes sketch trapezoids that show the trapezoidal sum is an overestimate.\vspace{6cm}
\item $\displaystyle \int_a^b f(x) \ dx$ is overestimated by both a right sum and by a mid-point sum. Sketch your function twice, on one set of axes sketch rectangles that show the right sum is an overestimate. On the other set of axes sketch trapezoids that show the midsum is an overestimate.
\end{enumerate}
\end{enumerate}
\end{document}